\documentclass[12pt]{article}
\usepackage[]{cite}
\usepackage{cmap}
\usepackage[T2A]{fontenc}
\usepackage[utf8]{inputenc}
\usepackage[english, russian]{babel}
\usepackage{amsmath, amsfonts,amssymb}
\usepackage{graphicx, epsfig}
\usepackage{subfig}
\usepackage{color}
\usepackage{enumitem}


\newcommand\argmin{\mathop{\arg\min}}
\newcommand{\T}{^{\text{\tiny\sffamily\upshape\mdseries T}}}
\newcommand{\hchi}{\hat{\boldsymbol{\chi}}}
\newcommand{\hphi}{\hat{\boldsymbol{\varphi}}}
\newcommand{\bchi}{\boldsymbol{\chi}}
\newcommand{\A}{\mathcal{A}}
\newcommand{\B}{\mathcal{B}}
\newcommand{\x}{\mathbf{x}}
\newcommand{\hx}{\hat{x}}
\newcommand{\hy}{\hat{y}}
\newcommand{\M}{\mathcal{M}}
\newcommand{\N}{\mathcal{N}}
\newcommand{\R}{\mathbb{R}}
\newcommand{\p}{p(\cdot)}
\newcommand{\q}{q(\cdot)}
\newcommand{\uu}{\mathbf{u}}
\newcommand{\vv}{\mathbf{v}}


\renewcommand{\baselinestretch}{1}


\newtheorem{Th}{Теорема}
\newtheorem{Def}{Определение}
\newenvironment{Proof} % имя окружения
{\par\noindent{\bf Доказательство.}} % команды для \begin
{\hfill$\scriptstyle\blacksquare$} % команды для \end
\newtheorem{Assumption}{Предположение}
\newtheorem{Corollary}{Следствие}
\newtheorem{problem}{Problem}


\textheight=24cm % высота текста
\textwidth=16cm % ширина текста
\oddsidemargin=0pt % отступ от левого края
\topmargin=-1.5cm % отступ от верхнего края
\parindent=24pt % абзацный отступ
\parskip=0pt % интервал между абзацами
\tolerance=2000 % терпимость к "жидким" строкам
\flushbottom % выравнивание высоты страниц

%\graphicspath{ {fig/} }



\begin{document}
	
	\thispagestyle{empty}
	\begin{center}
		\sc
		Министерство образования и науки Российской Федерации\\
		Московский физико-технический институт
		{\rm(государственный университет)}\\
		Физтех-школа прикладной математики и информатики\\
		Кафедра <<Интеллектуальные системы>>\\[35mm]
		\rm\large
		Владимиров Эдуард Анатольевич\\[10mm]
		\bf\Large
		Модели пространства состояний в задачах классификации сигналов ЭКоГ \\[10mm]
		\rm\normalsize
		010990 --- Интеллектуальный анализ данных\\[10mm]
		\sc
		Выпускная квалификационная работа бакалавра\\[10mm]
	\end{center}
	\hfill\parbox{80mm}{
		\begin{flushleft}
			\bf
			Научный руководитель:\\
			\rm
			д.~ф.-м.~н. Стрижов Вадим Викторович\\[5cm]
		\end{flushleft}
	}
	\begin{center}
		Москва\\
		2023
	\end{center}
	
	
	\newpage
	\tableofcontents
	\newpage
	
	\begin{abstract}
		
		Одной из самых популярных парадигм для построения нейрокомпьютерных интерфейсов являются потенциалы P300. В работе рассмотрена задача мультиклассовой классификации потенциалов, вызванных визуальными стимулами в игре в виртуальной реальности. Требуется, чтобы классификатор основывался на минимально возможном числе предъявлений стимулов, имел высокую предсказательную способность и не требовал калибровки на новых участниках. Предлагаются алгоритмы мультиклассовой классификации и методы улучшения качества предсказаний на основе аугментаций. Рассмотрены классические методы аугментаций и порождение выборки байесовскими моделями. Проведено сравнение нескольких алгоритмов предсказания целевых стимулов и демонстрируется (?), что предложенная схема обработки ЭЭГ позволяет получать предсказания более высокого качества, чем существующие алгоритмы.
		
		\bigskip
		\textbf{Ключевые слова}: \emph{ЭЭГ, P300, аугментация, байесовские модели, мультиклассовая классификация}
	\end{abstract}
	
	\newpage
	
	%%%%%%%%%%%%%%%%%%%%%%%%%%%%%%%%%%%%%%%%%%%%%%%%%%%%%%%%%%%%%%%%%%%%%%%%%%%%%%%%%%%%%%%%%%%%%%%%%%%%%%%%%%%%%%%%%%%%%%%%%%%%%%%%%%
	\section*{Введение}
	\addcontentsline{toc}{section}{\protect\numberline{}Введение}
	
	\paragraph{Актуальность темы.}
	Магистерская работа посвящена задаче мультиклассовой классификации потенциалов головного мозга P300. 
	%определения объекта внимания человека (выбор визуального стимула) по электроэнцефалограмме с выделенными в ней потенциалами P300.
	Нейрокомпьютерный интерфейс (НКИ) ~--- технология, позволяющая человеку отдавать команды компьютеру с помощью анализа данных о работе мозга. % без использования мышц. 
	В первую очередь нейрокомпьютерные интерфейсы разрабатывались для пациентов с ограниченными возможностями. НКИ используются для коммуникации пациентов с различными формами паралича и для реабилитации моторной функции после инсультов и спинальных травм \cite{chaudhary2020neuropsychological}. Для построения ИМК используются различные виды биосигналов как требующих операции для имплантации электродов \cite{collinger2013high}, так и неинвазивных:  fNIRS \cite{nazeer2020enhancing}, фМРТ \cite{yoo2004brain}, ЭЭГ. Нейроинтерфейсы на основе электроэнцефалограммы (ЭЭГ) остаются на данный момент наиболее популярными за счет лёгкой настройки, низкой стоимости и высокого временного разрешения метода. 
	Реактивные нейроинтерфейсы работают за счет выяления реакций мозга на специфическую стимуляцию. Пользователь привлекает свое внимание к одному из нескольких стимулов, которые вызывают реакции ЭЭГ. Стимулы могут иметь различную модальность (к примеру, слуховую и тактильную), но самые надежные ИМК создаются на основе зрительных стимулов разных типов, которые вызывают различные ответы мозга.
	%(ЗВПУС, КЗВП, P300). Визуальные стимулы на основе зрительных вызванных потенциалов стабильного состояния (ЗВПУС) мигают с определенной частотой, которая при фиксации взгляда на стимуле детектируется в зрительной коре спектральными либо корреляционными методами.  В ИМК на основе КЗВП по тому же принципу работают различные псевдослучайные последовательности \cite{bin2011high}. Количество возможных команд в этом типе интерфейса практически не ограничено, а точность выбора превышает 90\% с затратами времени несколько секунд на команду \cite{chen2014hybrid}. Таким образом, можно создать полноценную систему набора текста парализованным человеком. К недостатком метода относится сильная зависимость от активного внимания, выражающегося в фиксации взгляда на стимулах, а не в осознавании задачи \cite{davidson2020ssvep}. По сути, метод при сильном усложнении процедуры дает сопоставимые с отслеживанием движений глаз результаты. Кроме того, постоянное мигание на экране утомляет пользователя.
	Самый популярный вид реактивных ИМК ~--- ИМК на основе вызванного потенциала P300. В данном типе нейроинтерфейса управление совершается за счет привлечения внимания к разнесенным во времени стимулам. При активации стимула, на который пользователь хочет отреагировать, в окружении стимулов, которые ему безразличны, в центральных отведениях регистрируется потенциал, связанный с событием с латентностью (задержкой во времени) пика около 300 мс после стимула \cite{polich2007updating}. Нейроинтерфейсы на основе P300 имеют высокую скорость передачи информации за счет высокой точности классификации команд и потенциально большое их количество. На основе этой технологии также можно создание систем набора текста. Преимущество ИМК-Р300 заключается в меньшей чувствительности к модальности стимула - по сути, один и тот же интерфейс можно использовать со стимулами любых форм и размеров, при условии, что они достаточно легко различимы. Это позволяет создать интерфейс не только полезный пациентам, но и интересный для здоровых пользователей. На сегодняшний день существует множество подходов к использованию ИМК, основанных на неинвазивных электродах, для здоровых людей \cite{NER_2015}. Кроме того, ИМК могут быть использованы в рекреационных целях, например, в играх \cite{Recreational_Applications}, \cite{Kaplan_Shishkin_games}.
	
	% TODO: expand applications
	Р300 является универсальным потенциалом, так как он не требует большого количества электродов, его легко детектировать и достаточно просто им управлять \cite{from_sasha1}.  Данные потенциалы достаточно давно и успешно используются в управлении экзоскелетами \cite{fromsasha2} и устройствами для контакта с окружающим миром (устройство ввода символов, как клавиатура) для людей, восстанавливающихся после инсульта, а также пациентов с плегиями и параличами \cite{formsasha3}. Помимо этого, информация о Р300 позволяет производить диагностику когнитивных функций по латентности, амплитуде и локализации потенциала\cite{fromsasha4}.
	
	Несмотря на широкое использование нейрокомпьютерных интерфейсов со множественным выбором ответа, решение задачи классификации основывается на агрегации бинарных предсказаний \cite{sth_binary}.
	
	
	
	\paragraph{Цель работы.}
	Целью работы является разработка метода предсказания метки визуального стимула:
	\begin{itemize}
		\item применимого для различных людей без дополнительной настройки;
		\item позволяющего вносить корректировки при низкой уверенности классификатора в ответе;
		\item применимого для других парадигм работы с ээг (?), например, motor imagery;
		\item имеющего высокую точность.
	\end{itemize}
	И демонстрация этих свойств на практике.
	
	
	\paragraph{Методы исследования.}
	Для достижения поставленных целей используются аугментации для повышения качества, автокодировщики для построения признаковых пространств. Для сбора ЭЭГ использован BCI шлем Neiry. Для программной реализации использовался язык программирования Python.
	
	
	
	\paragraph{Основные положения, выносимые на защиту.}
	\begin{enumerate}
		\item Аугментация данных, дающая прирост точности
		\item Алгоритм оценки уверенности в ответе и возможности уточнения выбора ответа
		\item Решение прикладной задачи детектирования целевого стимула с использованием и без использования промежуточной бинарной классификации эпох
		\item Экспериментальное исследование разработанных алгоритмов, содержащее их сравнение с аналогами.
	\end{enumerate}
	
	
	
	\paragraph{Научная новизна.}
	Разработан новый подход к задаче мультиклассовой классификации потенциалов P300. На момент написания диссертации нет известных работ, не использующих промежуточную бинарную классификацию для определения целевых визуальных стимулов. Предложен подход к решению прикладной задачи, позволяющий с высокой точностью определять целевой стимул и предоставлять информацию о свойствах P300 для медицинских исследований в форме, понятной и интересной пользователю.
	
	
	\paragraph{Теоретическая значимость.}
	Использование латентного пространства автокодировщиков в качестве признакового описания ЭЭГ позволяет снизить размерность признакового пространства классификации и разделить тип P300 у исптыуемых.
	\paragraph{Практическая значимость.}
	Предложенные методы значительно улучшают пользовательский опыт использования парадигмы P300.
	А ткже позволют  исопльзовать простые, понятные и интересные игры в медицинских тестированиях.
	
	
	\paragraph{Степень достоверности и апробация работы.}
	Достоверность результатов подтверждена экспериментальной проверкой полученных методов на реальных задачах.
	%, публикациями результатов исследования в рецензируемых научных изданиях, в том числе рекомендованными ВАК. 
	Результаты работы докладывались и обсуждались на следующих научных конференциях
	\begin{itemize}
		\item 13-я Международная конференция «Интеллектуализация обработки информации», 2020 г.\cite{IDP2020};
		\item 63-я Всероссийская научная конференция МФТИ, 2020 г. \cite{mipt2020};
	\end{itemize}
	
	
	
	% \paragraph{Публикации по теме.}
	%Основные результаты по теме магистерской работы изложены в изданиях из списка ВАК \cite{medvednikova2012algorithm, kuznetsov2012algorithm, medvednikova2013construction, stenina2014reconciliation, gazizullina2015forecasting, stenina2015reconciliation}, двух сборниках докладов конференций \cite{stenina2014mipt, stenina2015lomonosov} и других печатных изданиях \cite{medvednikova2012pca, Stenina2015ordinal}.
	
	
	
	
	
	%%%%%%%%%%%%%%%%%%%%%%%%%%%%%%%%%%%%%%%%%%%%%%%%%%%%%%%%%%%%%%%%%%%%%%%%%%%%%%%%%%%%%%%%%%%%%%%%%%%%%%%%%%%%%%%%%%%%%%%%%%%%%%%%
	\newpage
	\section*{Обозначения}
	\addcontentsline{toc}{section}{\protect\numberline{}Обозначения}
	
	\begin{itemize}
		\item Эпоха ~--- секундный отрезок ЭЭГ, отсчитывающийся после активации стимула.
		\item Раунд (блок) ~--- последовательность эпох с общим целевым стимулом, где каждый стимул активируется один раз, порождая одну эпоху. Активация стимулов происходит в случайном порядке.
		
		\item Акт ~--- последовательность раундов с общим целевым стимулом, после которого принимается решение о целевом стимуле. Количество актов в экспериментах ~--- 5 при обучении и 10 при игре.
		
		\item Запись (игра) ~--- последовательность актов, полученная от одного человека за одну сессию записи (одну игру).
		
		\item Датасет ~--- коллекция записей, сделанных на одном и том же оборудовании с одинаковыми визуальными активациями.
	\end{itemize}
	
	
	
	
	
	
	%%%%%%%%%%%%%%%%%%%%%%%%%%%%%%%%%%%%%%%%%%%%%%%%%%%%%%%%%%%%%%%%%%%%%%%%%%%%%%%%%%%%%%%%%%%%%%%%%%%%%%%%%%%%%%%%%%%%%%%%%%%%%%%
	\newpage
	\section{Постановка задачи}
	
	Обычно в рассматриваемой парадигме решается две различных, но связанных задачи классификации:
	
	\begin{enumerate}
		\item \textbf{Бинарная}
		
		Задача определить, есть ли в рассматриваемой эпохе потенциал P300: классификация на целевые и нецелевые эпохи. 
		\begin{problem}
			Бинарная.
			
			Пусть $N$ ~---  количество каналов записи ЭЭГ, $T$ ~---  число отсчётов времени в одной эпохе.
			
			Входные данные ~--- 
			эпохи ЭЭГ: $X_b \in \R^{N \times T}$\\
			Целевая переменная: $ Y_b \in \{0, 1\} $\\
			Целевая функция: $ f_b: X_b \to Y_b$
		\end{problem}
		В этом случае каждая эпоха рассматривается как отдельный объект и классифицируется независимо от других. Такой формат подходит для тренировки в реальном времени, обучения модели на каждом человеке. В данной задаче как обучающая, так и тестовая выборки являются несбалансированными, т.к. каждый стимул активируется одинаковое количество раз, но только один из них является целевым. Таким образом, получается соотношение классов  $1$ к $(s-1)$, где $s$ ~--- количество стимулов.
		Основная метрика для этой задачи ~-- F1 мера. 
		%Точность (accuracy) для данной задачи не является достаточной метрикой из-за дисбаланса классов, поэтому в экспериментах также измерялись такие метрики как precision, recall, f1 and ROC AUC.
		
		\item \textbf{Мультиклассовая}
		Обычно в парадигме P300 представлено несколько стимулов (в рассматриваемом датасете ~--- 5). Задача определить стимул, выбранный пользователем. 
		\begin{problem}
			Пусть M ~--- количество предъявлений визуальных стимулов в одном акте игры, s ~--- число визуальных стимулов.
			
			Входные данные ~--- акт: $X_m \in \R^{M \times N \times T}$\\
			Целевая переменная: $ Y_m \in [0, 1 \dots s-1] $ \\
			Целевая функция: $ f_m: X_m \to Y_m$
		\end{problem}
		
		Здесь несколько эпох рассматриваются одновременно. В заданных условиях каждый стимул генерирует одинаковое количество эпох. Только один стимул может быть целевым, следовательно, остальные ~---  пустые. В данном случае классы более сбалансированы, так как каждый раз пользователь выбирает новый стимул.
		
		На данный момент мультиклассовая задача решается как агрегация бинарных задач, т.е.:
		$$P^{mult}_j = \frac{\sum_{i=1}^NP^{bin}_{j,i}}{N}$$
		$j$ ~--- рассматриваемый класс мультиклассовой классификации\\
		$N$ ~--- количество предъявлений одного стимула в акте игры\\
		$P_{j, i}^{bin}$ ~--- вероятность того, что стимул $j$ целевой при предъявлении $i$.\\
		
		Итоговый стимул выбирается как стимул с наибольшей вероятностью $P^{mult}_j$. В качестве метрики для данной задачи выбрана точность (accuracy).
		%$$accuracy= \frac{1}{N}\sum[y_{pred} = y_{true}]$$
		%As a standard approach we consider taking epochs grouped by activated stimulus, predicting their probabilities to be target and sum these probabilities. This way we obtain \emph{"activation score"} for each stimulus. Then multiclass prediction consists in choosing the stimulus with the highest activation score. We call it baseline multiclass classification.
		
	\end{enumerate}
	%%%%%%%%%%%%%%%%%%%%%%%%%%%%%%%%%%%%%%%%%%%%%%%%%%%%%%%%%%%%%%%%%%%%%%%%%%%%%%%%%%%%%%%%%%%%%%%%%%%%%%%%%%%%%%%%%%%%%%%%%%%%%%%%%%%%%%%
	\newpage
	\section{Обзор существующих алгоритмов }
	\subsection{Классификация}
	Основная задача данной работы ~--- повышение качества классификации при определении объекта внимания человека на основе потенциалов P300 в электроэнцефалограмме. 
	По данным ЭЭГ, после их предобработки (см. раздел 4), проводится мультиклассовая классификация для определения целевого визуального стимула. 
	
	На данный момент решение задачи мультиклассовой классификации основывается на агрегированных результатах бинарной классификации, где бинарная классификация определяет наличие потенциала P300 на отрезке ЭЭГ.
	%классификатор обучается на каждом человеке заново.
	При этом потенциалы P300 рассматриваются в парадигме, где только один стимул может быть целевым, а общее количество стимулов обычно не менее трёх.  Таким образом в бинарной задаче присутствует дисбаланс классов.  
	
	\begin{enumerate}
		\item LR, логистическая регрессия \cite{scikit-learn}
		
		Простая линейная модель для решения задачи классификации с l2-регуляризацией.\\
		В данном классификаторе варьируется константа регуляризации.
		
		
		\item SVM, классификатор на методе опорных векторов \cite{scikit-learn}. 
		
		Модель для классификации с l2-регуляризацией. Алгоритм находит опорные векторы каждого класса, максимизируя расстояние между ними. Гиперплоскость между опорными векторами ~--- граница принятия решения.\\
		В качестве ядра рассматриваются линейная и радиально-базисная функции. Также варьируется константа регуляризации.
		
		
		\item LDA, линейный дискриминантный анализ.
		
		Линейная классификация в пердположениях, что каждый класс распределён нормально и все классы имеют одинаковую матрицу ковариации.\cite{scikit-learn}
		
		\item ERPCov TS LR
		
		По данным ЭЭГ строятся матрицы ковариаций событийных потенциалов (event-related potentials, ERP) \cite{PyRiemann}. Полученные матрицы проецируются в касательное пространство (TS) \cite{PyRiemann}, после чего проводится классификация с помощью логистической регрессии.\\
		Варьируются тип оценочной функции для ERP ковариаций и константа регуляризации в логистической регрессии.
		
		\item Xdawn LDA
		
		Пространственная фильтрация сигнала алгоритмом XDawn, улучшающим соотношение сигнал/шум событийных потенциалов \cite{PyRiemann}. После фильтрации данные классифицируются алгоритмом LDA.\\
		Варьируется количество компонент декомпозиции ЭЭГ сигнала в алгоритме XDawn.
		
		
		\item ERPCov MDM
		
		Обработка данных спомощью ERP ковариаций. Классификация с помощью Minimum Distance to Mean классификатора.
		Классификация происходит по ближайшему центроиду, оцениваемому для каждого класса \cite{PyRiemann}. \\
		Варьируется тип оценочной функции для матриц ковариации.
		
	\end{enumerate}
	
	\subsection{Аугментация}
	%\subsubsection{Бинарная}
	%TODO: добавить наши результаты из статьи и с конференции
	Для решения проблемы дисбаланса классов к стандартным шагам препроцессинга ЭЭГ (децимация, фильтрация, ресэмплинг, клиппинг и нормировка) предлагается добавить аугментацию данных. Данный шаг позволяет сбалансировать классы для обучения бинарного классификатора и повысить качество как бинарной, так и мультиклассовой классификации.
	
	В работах, связанных с классификацией ЭЭГ, применяются различные методы аугментации данных. 
	В статье \cite{Krell2017} используются пространственные аугментации: варьируется расположение электродов. Авторы моделируют сигнал электроэнцефалограммы в пространстве (на голове) и с помощью интерполяций "сдвигают"\  электроды, получая измененную ЭЭГ. Для выборки из шести человек получен прирост мультикласовой точности около 2.5\%. Аугментации с помощью добавления шума часто используются во многих областях. В работе \cite{Zhang2018} авторы накладывают гауссовский шум на амплитуды сигнала после преобразования Фурье, затем обратным преобразованием Фурье возвращаются к исходному формату данных. Такой подход позволил получить прирост точности в 2.3\% для задачи представления движения при исследовании выборки из девяти человек. 
	
	В данной работе рассматривается два метода аугментаций. Назовём эпохой отрезок времени, отсчитываемый от момента предъявления визуального стимула, где предполагается наличие стимула P300. Тогда первый вариант аугментации данных ~--- изменение данных ЭЭГ в эпохе с помощью алгоритмов, основанных на SMOTE (synthetic minority over-sampling technique) \cite{smote}. Второй ~--- смещение точки отсчёта старта эпох. %Использование аугментаций позволяет повысить качество мультиклассовой классификации на величину до 6\%. 
	SMOTE-аугментации применительно к задаче классификации P300 были рассмотрены в работе \cite{Lee2020}. Использовалась выборка из 44 человек и две вариации алгоритма: SMOTE и borderline SMOTE. Этот подход обеспечил 1.3\% прироста точности в среднем. Но для людей, изначально имевших точность ниже 81\%, прирост составил 5.4\%. Аугментации при помощи небольших сдвигов стимулов относительно ЭЭГ по времени рассмотрены в работе \cite{Krell2018DataAF}  на небольшой выборке в пять человек. В результате этого подхода улучшение мультиклассовой точности составило ~1\%. В нашей работе использование подобных аугментаций обеспечивает прирост точности до 6\% и решает проблему дисбаланса классов, т.к. рассматривается несколько сдвигов для каждого целевого стимула в тренировочном датасете.
	
	%\subsubsection{Мультиклассовая}
	%Такая задача не решается и, соответственно, не аугментируется
	%%%%%%%%%%%%%%%%%%%%%%%%%%%%%%%%%%%%%%%%%%%%%%%%%%%%%%%%%%%%%%%%%%%%%%%%%%%%%%%%%%%%%%%%%%%%%%%%%%%%%%%%%%%%%%%%%%%%%%%%%%%%%%%%%%%%%%%
	\newpage
	\section{Предлагаемый метод}
	
	
	\subsection{Анализ недостатков существующих алгоритмов}
	
	
	\subsection{Наш подход}
	
	\subsection{Оценка качества}
	
	
	\subsection{Дополнительные способы повышения качества}
	\subsubsection{Аугментация}
	\subsubsection{Уверенность классификатора}
	
	
	
	%%%%%%%%%%%%%%%%%%%%%%%%%%%%%%%%%%%%%%%%%%%%%%%%%%%%%%%%%%%%%%%%%%%%%%%%%%%%%%%%%%%%%%%%%%%%%%%%%%%%%%%%%%%%%%%%%%%%%%%%%%%%%%%%
	\newpage
	\section{Вычислительный эксперимент}
	
	
	
	\subsection{Экспериментальные данные}
	\subsubsection{Наборы данных}
	\subsection{Предобработка данных}
	Перед решением задач классификации сигнал ЭЭГ должен пройти предобработку: препроцессинг всего сигнала и его разделение на эпохи.
	Используемые шаги препроцессинга стандартны и взяты из лучших практик по обработке сигналов:
	\begin{enumerate}
		\item Децимация со сглаживающим фильтром
		\item Полосовой фильтр Баттерворта
		\item Передискретизация сигнала с помощью линейной интерполяции
		\item Клиппинг зашкаливающих значений
		\item Поканальная нормировка (вычитание среднего и деление на стандартное отклонение всех значений каждого канала)
	\end{enumerate}
	
	Для обработки сигналов использовалась библиотека SciPy \cite{SciPy}.
	
	
	
	\subsubsection{Результаты}
	
	%%%%%%%%%%%%%%%%%%%%%%%%%%%%%%%%%%%%%%%%%%%%%%%%%%%%%%%%%%%%%%%%%%%%%%%%%%%%%%%%%%%%%%%%%%%%%%%%%%%%%%%%%%%%%%%%%%%%%%%%%%%%%%%%%%%%%%%
	\newpage
	\section*{Заключение}
	\addcontentsline{toc}{section}{\protect\numberline{}Заключение}
	
	
	%%%%%%%%%%%%%%%%%%%%%%%%%%%%%%%%%%%%%%%%%%%%%%%%%%%%%%%%%%%%%%%%%%%%%%%%%
	\newpage
	\addcontentsline{toc}{section}{\protect\numberline{}Список литературы}
	\bibliographystyle{ugost2008}
	\bibliography{cite, references,conference,other}
	
\end{document} 